\section{Introduction}

The study of gaps between consecutive prime numbers has been a central topic in number theory since antiquity. Let $p_n$ denote the $n$-th prime number and define the prime gap $\pg(n) = p_{n+1} - p_n$. The distribution of these gaps remains one of the most fascinating and challenging areas of investigation in modern number theory.

Recent work by Banks et al.\ \cite{banks2019} has suggested new approaches to understanding the statistical behavior of prime gaps, while computational evidence gathered by Oliveira e Silva et al.\ \cite{silva2014} has provided extensive numerical data supporting various conjectures about their distribution.

In this paper, we present three main contributions:
\begin{enumerate}
    \item A new upper bound for the maximal prime gap function $\maxgap(x)$
    \item Improved estimates for the average gap size in short intervals
    \item Novel applications to computational prime number theory
\end{enumerate}

Our results build upon the foundational work of Cramér \cite{cramer1936} while incorporating modern techniques from analytic number theory.