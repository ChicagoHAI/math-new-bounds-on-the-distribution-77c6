\section{Preliminaries}

\begin{definition}
For $x > 0$, the maximal prime gap function $\maxgap(x)$ is defined as:
\[\maxgap(x) = \max\{p_{n+1} - p_n : p_n \leq x\}\]
where $p_n$ denotes the $n$-th prime number.
\end{definition}

\begin{definition}
The average prime gap function $\avgap(x,h)$ over an interval $[x,x+h]$ is defined as:
\[\avgap(x,h) = \frac{1}{\pi(x+h) - \pi(x)} \sum_{x < p_n \leq x+h} \pg(n)\]
where $\pi(x)$ is the prime counting function.
\end{definition}

\begin{lemma}\label{lem:basic_bound}
For all $x \geq 2$, we have:
\[\maxgap(x) \geq \log x\]
\end{lemma}

\begin{proof}
This follows from the fact that there must be a composite number between any prime $p$ and $2p$, creating a gap of at least $\log p$.
\end{proof}